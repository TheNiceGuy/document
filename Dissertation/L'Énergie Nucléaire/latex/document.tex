\documentclass[11pt]{article}
\usepackage{standalone}
\usepackage[francais]{babel}
\usepackage[utf8]{inputenc}
\usepackage{marginnote}
\usepackage{tikz}
\usepackage{geometry}
\usepackage{setspace}
\usepackage[backend=bibtex,style=numeric-comp]{biblatex}
\usepackage[nodayofweek,level]{datetime}

\bibliography{bibliography}
\renewcommand{\baselinestretch}{2}
\geometry{
	total={210mm,297mm},
	left=34mm,
	right=34mm,
	top=34mm,
	bottom=34mm,
}

\newcommand{\mots}[1]{
  $\backslash\!\!\backslash$
  \marginpar{
    \raggedright\footnotesize\hfill
    \setstretch{1.025}#1 mots
  }
}

\begin{document}
\sloppy
\begin{titlepage}
    \setstretch{1.0}
	\centering
	\documentclass{standalone}
\usepackage{tikz}

\begin{document}
\definecolor{c0c1867}{RGB}{12,24,103}
\definecolor{ccf6003}{RGB}{207,96,3}

\begin{tikzpicture}[y=0.80pt, x=0.80pt, yscale=-1.000000, xscale=1.000000, inner sep=0pt, outer sep=0pt]
\path[fill=c0c1867] (29.6938,45.4250) .. controls (37.1578,34.3740) and
  (46.0118,32.7910) .. (53.0378,32.9290) .. controls (56.2258,32.7840) and
  (60.8368,33.7690) .. (66.3828,34.4720) .. controls (70.5778,35.0000) and
  (74.8778,35.7560) .. (79.8628,36.0780) .. controls (85.1368,36.4200) and
  (92.7098,35.7770) .. (99.1078,33.2360) .. controls (108.0508,29.6880) and
  (112.1298,26.3000) .. (117.0178,20.4720) .. controls (120.1758,16.7600) and
  (122.2368,12.8470) .. (123.0918,10.3550) .. controls (124.1588,7.0550) and
  (124.8638,3.6260) .. (124.6098,1.0000) .. controls (123.8498,4.0050) and
  (121.3208,6.7980) .. (119.9858,8.2630) .. controls (114.5898,14.3410) and
  (105.7028,18.8300) .. (97.1288,21.1300) .. controls (90.5008,22.9110) and
  (85.6728,23.3250) .. (78.9468,22.7120) .. controls (76.3888,22.4890) and
  (73.3288,21.9920) .. (69.4978,21.4450) .. controls (67.9768,21.2260) and
  (65.4018,20.8820) .. (63.7778,20.6590) .. controls (61.8978,20.4130) and
  (60.3528,20.2390) .. (58.7928,20.1050) .. controls (56.4588,19.9570) and
  (53.5698,19.9270) .. (50.8768,20.3830) .. controls (47.0308,21.0350) and
  (43.2858,22.5670) .. (39.9638,25.3560) .. controls (37.7528,27.2150) and
  (34.7188,30.9480) .. (33.2228,34.1060) .. controls (30.5508,39.6570) and
  (29.9888,45.3830) .. (29.6938,45.4250);
\path[fill=c0c1867] (48.5438,43.1690) .. controls (51.3408,42.9810) and
  (54.2718,43.3690) .. (57.4208,43.7680) .. controls (59.8498,44.1440) and
  (62.9378,44.5580) .. (65.9148,44.9610) .. controls (69.9488,45.5870) and
  (71.7658,45.7510) .. (74.4318,46.0260) .. controls (75.8158,46.1660) and
  (77.5468,46.4270) .. (78.8848,46.4590) .. controls (81.0218,46.5100) and
  (81.9278,46.3510) .. (83.2008,46.2950) .. controls (89.1668,45.5590) and
  (98.0738,43.1160) .. (102.0088,34.7670) .. controls (101.3428,41.1000) and
  (98.9928,44.7310) .. (96.9558,47.7300) .. controls (91.7878,55.3280) and
  (85.2598,58.5770) .. (77.9678,59.0690) .. controls (75.1698,59.2600) and
  (72.7868,59.0920) .. (69.6378,58.6920) .. controls (67.0798,58.2990) and
  (64.5548,57.9640) .. (61.4078,57.5400) .. controls (58.5998,57.1620) and
  (55.4908,56.7400) .. (52.8248,56.3450) .. controls (51.4408,56.2070) and
  (50.2128,56.0330) .. (48.8778,55.9990) .. controls (46.7408,55.9500) and
  (45.2638,56.0270) .. (43.9898,56.0800) .. controls (38.0238,56.8160) and
  (31.2968,59.2980) .. (25.1798,67.6110) .. controls (27.0598,59.1110) and
  (28.4588,56.7020) .. (29.9948,54.4400) .. controls (35.1568,46.8360) and
  (41.2518,43.6640) .. (48.5438,43.1690);
\path[fill=ccf6003] (93.9068,75.0460) .. controls (93.5098,76.7910) and
  (92.4888,78.8940) .. (91.3258,80.3470) .. controls (86.6188,84.6790) and
  (72.8168,91.7000) .. (61.2418,93.4310) .. controls (47.5788,95.4720) and
  (34.5958,94.0250) .. (24.2168,89.9820) .. controls (25.8168,87.7900) and
  (29.3418,84.6610) .. (29.9958,84.2050) .. controls (39.5188,88.2210) and
  (52.0128,89.7150) .. (65.2748,87.7320) .. controls (76.4688,86.0590) and
  (86.3038,80.2040) .. (93.9068,75.0460);
\path[fill=ccf6003] (18.8998,87.5890) .. controls (9.2258,82.5290) and
  (2.6488,74.8930) .. (1.2678,65.6460) .. controls (-1.2472,48.8290) and
  (14.1968,32.0820) .. (37.3128,24.6490) .. controls (35.4898,26.1130) and
  (30.8888,31.2560) .. (29.4668,37.7290) .. controls (17.7578,44.9660) and
  (10.8918,55.1110) .. (12.4038,65.2050) .. controls (13.2378,70.8020) and
  (16.5378,75.6600) .. (21.5808,79.4770) .. controls (20.3848,81.7890) and
  (19.4298,84.9300) .. (18.8998,87.5890);
\path[fill=c0c1867] (43.5098,66.1520) .. controls (45.9208,65.9160) and
  (48.8088,66.3020) .. (51.0228,66.5490) .. controls (53.2348,66.8190) and
  (56.0548,67.3020) .. (58.0858,67.5420) .. controls (62.8418,68.1040) and
  (64.8618,68.6650) .. (69.1148,68.9360) .. controls (77.2498,68.7150) and
  (81.1438,65.9160) .. (83.8588,60.6120) .. controls (83.3398,65.9200) and
  (81.3838,68.5180) .. (79.9428,70.8200) .. controls (77.0718,74.8270) and
  (73.7138,77.1470) .. (69.3298,78.5150) .. controls (66.9878,79.4280) and
  (64.7728,79.8860) .. (61.0868,79.8890) .. controls (56.7978,79.8930) and
  (53.7768,79.0590) .. (50.2738,78.5860) .. controls (45.4748,77.9390) and
  (47.4598,78.1040) .. (40.8188,78.1040) .. controls (34.1168,78.1040) and
  (26.8858,82.1130) .. (20.3978,90.9670) .. controls (22.2508,82.5690) and
  (23.6508,80.1240) .. (26.5358,75.8420) .. controls (30.4938,70.3390) and
  (36.0778,66.8770) .. (43.5098,66.1520);
\path[fill=c0c1867] (56.6848,10.0080) .. controls (59.0708,8.8030) and
  (63.9748,6.2450) .. (68.4338,4.5620) .. controls (77.9938,0.9520) and
  (85.5358,0.1140) .. (89.8298,2.4550) .. controls (98.5508,7.2070) and
  (87.7768,17.3620) .. (86.0438,18.7230) .. controls (86.6338,17.7150) and
  (86.4018,16.4010) .. (86.2248,15.8320) .. controls (84.9228,12.5680) and
  (80.5728,11.0610) .. (77.6218,10.0870) .. controls (73.7228,8.8000) and
  (62.5298,8.3160) .. (56.6848,10.0080);

\end{tikzpicture}
\end{document}
\par\vspace{5mm}
	{\scshape\LARGE Cégep de Trois-Rivières\par}
	\vspace{15mm}
	{\scshape\Large Text Argumentatif\par}
	{\huge\bfseries L'énergie Nucléaire de Demain\par}
	\vspace{30mm}
	{\Large\itshape Gabriel-Andrew Pollo-Guilbert\par}
	\vfill
	Travail remis à\par
	Samuel \textsc{Sénéchal}
	\vfill
	\formatdate{11}{5}{2016}\par
	824 mots
\end{titlepage}

L'évolution de la société promet, en général, une meilleure qualité de vie à la population. Par contre, cette promesse a ses coûts. Afin de satisfaire les besoins de la société d'aujourd'hui, la croissance exponentielle de la demande énergétique émet une grande quantité de gaz à effet de serre. La répercussion\mots{50}majeure de ces gaz est le réchauffement climatique. Ce problème peut être remédié en diminuant les sources énergétiques fossiles, comme le pétrole ou le charbon, en faveur des énergies propres. La société a le choix entre investir dans l'amélioration des énergies renouvelables, comme l'énergie solaire ou éolienne, ou investir dans\mots{100}l'énergie nucléaire qui est moins acceptée du grand public. Étant donnée la nature controversée de l'énergie nucléaire, est-ce qu'il faudrait investir dans l'énergie nucléaire de demain? L'avancement dans l'énergie nucléaire devrait être encouragé, car il permettrait à cette source d'énergie d'être économiquement compétitif tout en développant de nouveaux marchés pour\mots{150}cette énergie.
 
Pour commencer, le développement de l'énergie nucléaire réduirait les coûts reliés à cette énergie. Les réacteurs de demain sont si différents qu'ils ne requièrent pas autant de réglementations par le gouvernement. En effet, les régulations d'aujourd'hui sont si sévères qu'elles occupent la majorité des capitaux nécessaires d'un tel\mots{200}projet. La flotte actuelle de réacteurs est composée principalement de réacteurs à eau pressurisée qui doivent opérer à de très grandes pressions. Ce n'est pas sans risque. Dans l'évènement où l'eau sous pression s'échappe, il est difficile de capturer la vapeur, car son volume augmente par un facteur d'environ 1000.\mots{250}Pour contrer cet évènement peu probable, le gouvernement oblige la construction d'une gigantesque enceinte de confinement pour éviter une fuite d'éléments radioactifs. Ce bâtiment qui entoure un réacteur nucléaire est très couteux, car il requiert une grande quantité de béton de grade nucléaire. Les technologies de demain, comme le réacteur\mots{300}à sels fondus, favorisent l'opération du réacteur à une basse pression en remplaçant l'eau pressurisée par un autre liquide de refroidissement. Par conséquent, la construction de cette enceinte de confinement n'est pas nécessaire. Bref, ce n'est qu'une des régulations qui pourraient être levées afin de réduire les coûts. De plus,\mots{350}les réacteurs de demain sont beaucoup plus simples. Non seulement la simplicité par elle même réduit les coûts, mais elle permet aussi la conception d'une chaîne de montage pouvant construire des réacteurs en masse. En effet, plus on produit des réacteurs, moins le coût est élevé. Il est estimé que\mots{400}les règlements moins sévères ainsi que la production de masse des réacteurs réduisent le coût à environ 2 \$/w par réacteur\cite{IMSR}, ce qui est économiquement plus compétitif que les sources renouvelables.
 
Pour continuer, les avancées dans l'énergie atomique ouvrent aussi plusieurs autres possibilités. Par exemple, cette source d'énergie pourrait être\mots{450}utilisée dans d'autres domaines comme l'industrie ou l'armée. Pour l'instant, plusieurs procédés chimiques, comme la désalinisation de l'eau ou la production d'hydrogène, requièrent beaucoup d'énergie thermique qui provient de sources fossiles. En général, les prochains concepts de réacteurs nucléaires opèrent à très grande température et pourraient remplacer l'énergie fossile dans\mots{500}ces applications. Dans le domaine militaire, l'énergie utilisée par les bases dans les régions éloignées provient principalement d'énergies polluantes. L'armée américaine étudie la possibilité d'avoir un réacteur modulaire beaucoup plus compact et efficace pouvant être utilisé au champ de bataille\cite{military}. Cela éliminerait principalement les risques reliés à la logistique pour\mots{550}transporter le gaz ou le pétrole. Bref, le développement de cette énergie offre plus de possibilités que seulement la production d'énergie civile. De plus, les futurs réacteurs offrent aussi la possibilité de recycler les déchets radioactifs. En physique nucléaire, la théorie pour recycler les déchets est très bien connue. La\mots{600}meilleure méthode pour recycler des éléments radioactifs comme le plutonium est d'utiliser un réacteur à neutrons rapides. Le danger principal de ces réacteurs provient de l'utilisation du sodium liquide en tant que modérateur et liquide de refroidissement. La réactivité du sodium avec l'oxygène ou l'eau fut la source de plusieurs\mots{650}accidents dans des centrales expérimentales. Les technologies de demain cherchent à remplacer le sodium par de l'hélium gazeux ou par du plomb liquide, les deux étant beaucoup moins dangereux. En résumé, l'utilisation de l'énergie nucléaire a le potentiel d'être beaucoup plus vaste tout en diminuant la production et les réserves\mots{700}de déchets radioactifs.
 
En rappel, la question de l'investissement dans l'énergie nucléaire est pertinente avec le changement énergétique à venir. Il est préférable de continuer l'investissement, car les futures technologies nucléaires promettent d'être économiquement compétitives aux autres sources énergétiques tout en élargissant l'application de celles-ci. Malheureusement, il est difficile d'encourager\mots{750}ce chemin, car le public ne garde en mémoire que les terribles accidents nucléaires comme Fukushima Daiichi ou Tchernobyl. Il ne va pas sans dire que les médias jouent un rôle important dans l'influence à ce sujet. En effet, l'information ramenée au publique est souvent exagérée ou même incorrecte. Il\mots{800}serait intéressant de voir la répercussion s'ils étaient plus portés à instruire le public sur les sujets controversés comme celui-ci que de chercher le sensationnalisme.

\printbibliography
\end{document}