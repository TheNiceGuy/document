\documentclass[11pt]{article}
\usepackage[utf8]{inputenc}
\usepackage[frenchb]{babel}
\usepackage{tabularx}
\usepackage{macro}

\begin{document}
\definition{Empirisme}{C'est une théorie philosophique qui dicte que la connaissance provient des sens. Elle est opposé au rationnalisme.}
\definition{Hédonisme}{C'est une théorie philosophoque qui dicte que le but ultime de la vie est d'être heureux.}
Afin d'être le plus heureux possible, il faut augmenter le plaisirs et éviter la souffrance. La quête du plaisirs s'observe empiriquement chez toutes espèces ayant des capacités sensorielles. L'hédonisme est le critère central de l'utilitarisme.
\definition{Conséquentialisme}{C'est une théorie éthique qui dicte que les actions sont jugées en fonction des conséquences.}
Avec l'hédonisme, il est possible de vérifier si une conséquence amène le plaisirs ou la souffrance. Dans l'utilitarisme, l'intention est ignoré.
\definition{Dignité}{Chez les utilitarismes, elle provient de la capacité à ressentir le plaisirs ou la douleur}
Selon cette définition, les animaux sont importants dans la moralité d'une action. L'action bonne est celle qui a la plus grande «utilité» (apporte le plus de plaisirs). Enlever de la souffrance ou ajouter du plaisirs compte également.\\

Chez Jérémy Bentham, tout les plaisirs ont la même valeur, mais ils peuvent être quantifiés.

\tabletitle{Tableau 5.1}{7 critères du plaisirs}
\begin{center}
\begin{tabular}{@{}ll@{}}
\hline
  \textbf{Critère}
& \textbf{Importance du critère}\\
\hline
  Durée
& Plus le plaisirs est long\\
  Intensité
& Plus le plaisirs est intense\\
  Certitude
& Plus le plaisirs est certain d'être réalisé\\
  Proximité
& Plus le plaisirs est proche dans le temps\\
  Étendue
& Plus le plaisirs touche des personnes\\
  Fécondité
& Plus le plaisirs crée d'autres plaisirs\\
  Pureté
& Plus le plaisirs est pure (moins de douleur)\\
\hline
\end{tabular}
\end{center}

Avec ces critères, il est possible de déterminer qu'une action est meilleur qu'une autre en quantifiant le plaisirs. Lors de l'évaluation, il faut se détacher de notre bien-être (observateur impartial). Il faut être objectif, car tout le monde est considéré comme égaux.\pagebreak

Selon l'utilitarisme de l'acte, il faut suivre ces étapes.
\begin{enumerate}
\item Évaluer les différentes actions possibles
\item Évaluer les conséquences sur le bien-être commun de ces actions. Il effectuer le calcul d'utilité selon les critères les plus importants.
\item Choisir l'action qui fournit le plus grand plaisirs au plus grand nombre de personne.\\
\end{enumerate}

John Stuart Mills veut résoudre certain problème à l'éthique de Bentham énoncé ci-dessus.
\begin{itemize}
\item[$\bullet$] Il voit l'éthique de Bentham comme une éthique de pourceau, car tous les plaisirs, peu importe la nature vulgaire ou non.
\item[$\bullet$] Il trouve qu'il est encombrant de toujours faire un calcul d'utilité avant chaque action. De plus, le calcul peut s'avéré difficile.
\item[$\bullet$] Il amène aussi des situations où la bonne action semble immorale. \exemple{Le fait de sacrifier une personne inutile pour le bien de la majorité est un bonne action selon l'éthique de Bentham.}\\
\end{itemize}

Mills ajoute la qualitication des plaisirs. Pour lui, les plaisirs intellectuels sont supérieur à ceux du corps (plaisirs facile à atteindre). De plus, il défend que seul la personne qui sait apprécier les deux types de plaisirs peut les comparer. Le plaisirs plus important peut être définit par des «experts», mais il est disponible à quiquonque qui souhaite s'instruire.

\quote{Vaut mieu être un Socrate insatisfait qu'un imbécile satisfait.}{John Stuart Mills}
\definition{Laisser-Faire}{C'est un principe qui dicte que l'individu est toujours mieux placé pour savoir ce qui est bon pour lui.}
Mills ajoute ce principe du laisser-faire. Il en découle que l'individu n'a pas d'obligations morales envers lui. \exemple{La prostitution n'est pas nécéssairement une mauvaise chose.} L'état ne devrait pas punir un individu tant qu'il ne cause pas un tord direct aux autres.\\

Finalement, Mills ajoute les règles morales communes, ou plutôt, des calculs d'utilités pré-faits. Pour déterminer si une action est bonne, le calcul devrait être faite en fonction de cette règle. \exemple{Est-ce que la règle maximise le bien-être de la communauté?}\pagebreak

Selon l'utilitarisme de la règle, il faut suivre ces étapes.
\begin{enumerate}
\item Identifier quelle règle s'applique dans la situation donnée.
\item Évaluer si, globalement, le respect de cette règle apporte le bien-être commun. On applique le calcul d'utilité sur la règle.
\end{enumerate}

Si la règle contribue au bien-être commun, il faut s'y soumettre. Sinon, la règle peut être modifiée ou supprimée.\\

Finalement, le modèle de Mills est une sorte d'hybride entre Bentham et Kant. Il faut suivre les règles, mais des exceptions peuvent être faites dans les cas extraordinaires. Logiquement, il faut minimiser les exceptions, car elles affaiblissent la règle.
\end{document}