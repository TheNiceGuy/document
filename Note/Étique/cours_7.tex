\documentclass[11pt]{article}
\usepackage[utf8]{inputenc}
\usepackage[frenchb]{babel}
\usepackage{tabularx}
\usepackage{macro}
\usepackage{footnote}

\makesavenoteenv{tabular}

\begin{document}
\definition{Technique}{C'est un procédé définis et transmissible destiné à produire un résultat utile.}

\tabletitle{Tableau 7.1}{différence entre le technique rudimentaire et moderne}
\begin{center}
\begin{tabular}{@{}ccc@{}}
\hline
& \textbf{Rudimentaire}
& \textbf{Moderne}\\
\hline
  Effet
& immédiat
& à long terme\\
  Utilité
& survie humaine
& modification humaine\\
  Risque
& peu dangereux
& grand danger potentiel\\
\hline
\end{tabular}
\end{center}

\tabletitle{Tableau 7.1}{différence entre l'éthique classique et nouvelle}
\begin{center}
\begin{tabular}{@{}cc@{}}
\hline
  \textbf{Éthique Classique}
& \textbf{Éthique Nouvelle}\\
\hline
  limité à l'immédiat
& fait en fonction du futur\\
  anthropocentrique\footnote{Les animaux sauvages n'ont pas la capacité de respecter les droits, alors ils n'ont pas de droits. Le droit est réciproque.}
& centrée sur l'être\\
\hline
\end{tabular}
\end{center}

L'être humain domine la nature, donc il peut la plier à son vouloir. L'être humain se doit de considérer les autres. L'humain possède aujourd'hui le pouvoir de modifier sa propre vie. Il peut prolonger la vie, contrôler le comportement et contôler la génétique.

\quote{La promesse de la technique moderne s'est inversée en menace.}{Hans Jonas}

Selon Jonas, on dirait qu'on a perdu le contrôle sur la technologie. Le progrès ne peut pas être arrêter. On développe parce qu'on peut et non parce qu'on en a besoins. Le progrès technique fait voler en éclat nos points de repère éthiques.
\begin{itemize}  
\item La technique est devenue une fin en elle-même, alors qu'elle-même devrait seulement être un moyen.
\item Nous devons trouver un moyen de «reprendre le pouvoir sur notre pouvoir».
\end{itemize}

L'humain a une obligation morale de connaitre et de prévoir les conséquences à long terme de ses actions.
\begin{itemize}  
\item Les experts ont une responsabilité morale plus grande que les autres.
\item L'éthique doit prendre le relais de la connaissance incomplète.
\end{itemize}

La peur est une obligation morale pour Jonas, car elle doit nous inciter à la sagesse.
\begin{itemize}  
\item Il faut toujours évaluer les conséquences selon le pire scénario.
\item Être responsable implique d'accorder une grande importance au «scénario catastrophe».
\end{itemize}

\quote{Le reproche de «pessimisme» à l'addresse d'une telle partialité en faveur de la «prophétie de malheur» est réfutable par l'argument que le plus grand pessimisme est du côté de ceux qui tiennent le donné pour suffisamment mauvais ou non valable pour accepter n'importe quel risque au nom de son amélioration potentielle.}{Hans Jonas}

Jonas définit deux nouveaux impératifs catégoriques.
\quote{Agis de façon à ce que les effets de ton action soient compatibles avec la permanence d'une vie authentiquement humaine sur Terre.}{Hans Jonas}

\quote{Agis de façon à ce que les effets de ton action ne soient pas destructeurs pour survie indéfinie de l'humanité sur Terre.}{Hans Jonas}

Lorsque Jonas parle de «vie authentiquement humaine», il faut éviter tout ce qui :
\begin{itemize}  
\item défigure ou dégrade l'image de l'humanité.
\item déshumanise l'humain ou change son essence.
\item retire la capacité d'être libre, autonome ou responsable à l'humain.
\end{itemize}

\definition{Responsabilité Retrospective}{C'est «être responsable d'un événement déjà arrivé.»}
\definition{Responsabilité Prospective}{C'est «avoir des responsabilités, des obligations concernant l'avenir.»}

La responsabilité est un concept important pour Jonas. Il existe plusieurs modèles de responsabilité comme la responsabilité parentale et politique. Selon Jonas, il faut voir les nouvelles technologies d'un point de vue d'un parent. Il faut prendre en compte l'humanité et la Terre.
\begin{itemize}  
\item L'existence de l'enfant est une bonne chose.
\item On accepte de prendre en charge un fardeau lourd et imprévisible.
\item Nous restons parents toute notre vie.
\end{itemize}

Contrairement à la responsabilité parentale, la responsabilité politique est limitée dans le temps, mais elle fonctionne sous le même mode.
\begin{itemize}  
\item Le dirigeant doit viser le bien-être de la communauté à long terme.
\item Il doit être accepter d'être sollicité à n'importe quel moment.
\end{itemize}

Dans le modèle de Jonas, il faut suivre cette procédure d'évaluation afin de déterminer la moralité d'une technologie.
\begin{enumerate}
\item Il faut analyser la pratique ou la technologie.
\begin{enumerate}
  \item Connaître et prévoir les conséquiences à long terme de notre action
  \item En cas d'incertitude, il faut accorder une importance primordiale au scénario catastrophe.
\end{enumerate}
\item Il faut prendre une décision qui est conforme avec le nouvel impératif catégorique.
\end{enumerate}
\end{document}