\documentclass[11pt]{article}
\usepackage[utf8]{inputenc}
\usepackage[frenchb]{babel}
\usepackage{tabulary}
\usepackage{macro}

\begin{document}
En général, l'utilitarisme a certains problèmes. Entre autres, le calcul d'utilité peut être difficile à évaluer.
\begin{itemize}
\item Il peut manquer d'informations. \exemple{Si on ne se base pas sur les faits, la pratique de la simulation de noyade est une bonne action. Pourtant, elle s'avère inefficace.}
\item Les conséquences restent imprévisibles, on les anticipe.
\item Il y a absence d'intersubjectivité. On ne peut pas être sûre des sentiments des autres.
\item La existe différentes définitions du plaisirs. \exemple{Les douchbags n'ont pas les mêmes plaisirs que les moines.}
\end{itemize}

L'utilitarisme de Bentham a aussi des problèmes de plus.
\begin{itemize}
\item Il justifie certains actes contre-intuitifs.
\item Il y a une danger de «tyranie de la majorité». Les droits fondamentaux risquent d'être brimés au nom duplaisirs du plus grand nombre.
\item L'égalité des plaisirs amène une société de lâche.
\item Il y a un altruisme surhumain. \exemple{On devrait toujours donner de l'argent ou nos organes à ceux qui en ont besoins.}
\end{itemize}

L'utilitarisme de Mills a aussi ces problèmes. Il est surtout incohérent.
\begin{itemize}
\item Le fait de respecter une règle contraire à l'intêret de tous n'est pas l'utilitarisme. Mills fait un peu de «fétichisme».
\item Le fait de renoncer à une règle, car elle n'est pas à l'avantage de tous, c'est juger selon l'acte et non la règle.
\end{itemize}

Par contre, l'utilitarisme à certaines forces en général.
\begin{itemize}
\item C'est une éthique très simple à comprendre.
\item C'est une éthique flexible.
\item C'est une éthique ayant un caractère égalitaire et démocratique.
\end{itemize}

\tabletitle{Tableau 6.1}{Comparaison entre Kant et l'Utilitarisme}
\begin{center}
\begin{tabulary}{\textwidth}{@{}CCC@{}}
\hline
& \textbf{Kant}
& \textbf{Utilitarisme}\\
\hline
  Dignité morale
& Raison
& Sens\\
  Statut éthique
& Créatures rationnelles 
& Créatures sensorielles\\
  Moralité d'une action
& Conformité au devoir
& Maximier le plaisirs\\
  Obligation morale
& Oui
& Non\\
\hline
\end{tabulary}
\end{center}
\end{document}