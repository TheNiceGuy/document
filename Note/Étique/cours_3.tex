\documentclass[11pt]{article}
\usepackage[utf8]{inputenc}
\usepackage[frenchb]{babel}
\usepackage{tabularx}
\usepackage{macro}

\begin{document}
L'éthique chez Kant est basé sur deux idées fondamentales : la liberté et l'autonomie.

\definition{Liberté}{Pour Kant, la liberté provient des contraintes qu'un invidivu se donne. L'individu doit être autonome.}

En d'autres termes, une personne qui réalise tous ce qu'il veut n'est pas libre, car elle serait l'esclave de ses désirs. L'autonomie est importante, car Kant veut inciter le peuple à utiliser la raison afin d'arriver à des conclusions logiques sur ce qui est bien ou ce qui est mauvais. C'est par cette réflexion que plusieurs personnes peuvent arriver aux mêmes conclusions.\\

Pour Kant, le critère pour déterminer la moralité d'une action est l'\textbf{intention}. Les conséquences reliées à cette action sont totalement ignorées, car elles sont imprévisibles.

\definition{Bonne volonté}{C'est une action faite par devoir et non par inclinaison personnelle (action désintéressé). Plus l'écart entre l'intêret et le devoir est grand, plus l'action est admirable.}

Avoir une bonne volonté (intention), c'est le vouloir de se conformer au devoir moral. Par contre, avoir «bonne volonté» n'est pas un synonyme de «vouloir faire plaisirs». \exemple{L'action de voler pour payer des amis du luxe est mal.}\\

\tabletitle{Tableau 3.1}{moralité d'une action}
\begin{center}
\begin{tabular}{@{}lc@{}}
\hline
  \textbf{Exemple}
& \textbf{Moralité}\\
\hline
  dénoncer
& bien\\
  dénoncer pour se venger
& mal\\
\hline\\
\end{tabular}
\end{center}

Le devoir moral ne peut se fonder sur le \textbf{sentiment}. Les sentiments faussent l'action. En effet, c'est beaucoup plus facile d'aider un proche qu'un inconnue. Lorsque l'action est vide de sentiments, l'action est plus admirable.\\

D'une manière similaire, l'\textbf{humeur} est influencé par le monde extérieur. Ça n'influence pas la moralité des actions.\\

La base du devoir est la raison. Une fois qu'une chose est comprise, il est possible de trouver des vérités universelles. Selon Kant, un être qui est toujours rationnel est une être totalement bon. Hors, l'humain n'est pas parfaitement rationnel, donc il être guidé par une loi morale.

\definition{Impératif hypothétique}{C'est une proposition conditionnelle qui dicte qu'une action peut être bonne en vue de quelque chose. Elle est souvent de la forme suivante : «si tu veux X, tu dois faire Y».}

Cette proposition est souvent utilisée pour trouver le meilleur moyen d'atteindre un but. Par contre, elle n'est pas un devoir moral. Des fois, l'impératif hypothétique peut prendre une autre forme. Elle peut avoir la même forme grammaticalement que l'impératif catégorique.

\definition{Impératif catégorique}{C'est une proposition qui dicte ce qu'il faut faire. Elle est de la forme : «tu dois faire X».}

Contrairement à l'impératif hypothétique, cette proposition est un devoir moral. Elle ne dépend pas des circonstances, des désirs ou des buts d'une personne.

\definition{Loi morale}{C'est un impératif catégorique, une loi qui a de la valeur en soi et qui ne dépend pas des circonstances ou des objetifs.}

Kant donne 3 configurations différentes, mais équivalentes, de l'impératif catégorique.

\quote{Agis uniquement d'après la maxime qui fait que tu peux vouloir en même temps qu'elle devienne une loi universelle de la nature.}{Emmanuel Kant}

\definition{Maxime}{C'est une règle d'action subjectif donné à soi-même. Elle peut être morale ou immorale. Chacune de nos actions est guidée par une de ces règles.}

Selon cette première définition de Kant, une maxime n'est pas morale si :
\begin{itemize}
\item[$\bullet$] Son universalisation rend l'action impossible (contradiction logique).
\item[] \exemple{Si tout le monde se met à tricher, alors le concept de tricher serait impossible, car personne ne pourrait copier sur quelqu'un.}
\item[$\bullet$] Son universalisation entraine la destruction de l'humanité.
\item[$\bullet$] Son universalisation va à l'encontre des intêrets fondamentaux de toute personne rationnelle.
\end{itemize}

\quote{Agis de telle sorte que tu traites l'humanité aussi bien dans ta personne que dans la personne de tout autre toujours en même temps comme une fin, et jamais simplement comme un moyen.}{Emmanuel Kant}

\definition{Dignité}{Pour Kant, la dignité découle de la liberté. Puisqu'un humain est en mesure de déterminer ses propres buts (rationnalité), il mérite la considération.}

Kant sous-entend qu'il faut traité l'humanité comme soi-même avec dignité. Le mutualisme est possible, mais jamais seulement en avantage. Afin de traiter une personne avec dignité, il faut obtenir le conscentement libre et éclairé de la personne. Il faut aussi l'aider dans la mesure du possible si l'autre n'a pas un objetif immoral. Finalement, il faut respecter les droits fondamentaux de la personne.\\

Dans le royaume des fins de Kant, il existe deux respects. Le \textbf{respect négatif} implique qu'il faut se limiter. \exemple{Par respect pour les autres, je ne dois pas voler.} Le \textbf{respect positif} implique qu'il faut venir en aide. \exemple{Par respect pour les autres, je dois aider les plus démunis.}
\end{document}