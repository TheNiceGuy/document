\documentclass[11pt]{article}
\usepackage[utf8]{inputenc}
\usepackage[frenchb]{babel}
\usepackage{tabularx}
\usepackage{macro}

\begin{document}
\tabletitle{Tableau 2.1}{types de jugements}
\begin{center}
\begin{tabularx}{\textwidth}{@{}rX@{}}
\hline
  \textbf{Type de jugement}
& \textbf{Définition}\\
\hline
  Jugement de fait
& C'est une affirmation objective. Il est vérifiable.\\
  Jugement de goût
& C'est une affirmation subjective. Il est discutable.\\
  Jugement de valeur
& C'est une affirmation à la fois subjective et objective. Il fait appel à des valeurs. Il est discutable. Le déssacord peut être raisonné et il est possible de changer de position.\\
\hline\\
\end{tabularx}
\end{center}

Il est possible d'avoir un jugement de goût positif et un jugement de valeur négatif à l'égard d'une chose. \textit{Exemple : Il est possible d'aimer un artiste pourrit.}

\definition{Nihilisme moral}{C'est une position fondamentale qui dit que le bien et/ou le mal n'existent pas réellement. Il dicte que le jugement de goût est le même que le jugement de valeurs.}

Le problème avec cette position est qu'il peut être indiscutable. On ne peut pas justifier un jugement de valeur avec un jugement de goût. \exemple{Cet artiste est bon, car ce sont mes goûts personnels.}

\definition{Universalisme moral}{C'est une position fondamentale qui permet de donner un sens à la notion de progès moral. Il dicte qu'il est possible de trouver des réponses objectives. Donc, il permet l'existance de point de vue objectif.}

Le problème avec cette position est qu'il est difficile d'expliquer comment on découvre les faits moraux.

\definition{Relativisme moral}{C'est une position fondamentale similaire à l'universalisme moral. La différence est qu'il dicte que les vérités objectives sont relatifs au groupe.}

Même s'il favorise la tolérance, le problème avec cette position est qu'il est difficile de justifier la critique de sa propre culture (ou d'une autre). Le cadre de référence est aussi difficile à identifier.\pagebreak

\definition{Jugement moral}{C'est la justification d'une action selon une personne. Elle diffère selon le stade de développement de l'invididue.}

\tabletitle{Tableau 2.2}{niveau de développement}
\begin{center}
\begin{tabular}{@{}ccc@{}}
\hline
  \textbf{Période}
& \textbf{Niveau dév. cognitifs}
& \textbf{Niveau dév. moral}\\
\hline
  0-2 ans
& sensori-moteur
& \\
  2-6 ans
& préopératoire
& préconventionel\\
  6-12 ans
& opératoire
& conventionnel\\
  12+ ans
& formel
& postconventionnel\\
\hline\\
\end{tabular}
\end{center}

Les stades sont séquentiels, irréversibles, intégratifs, transculturels et mènent à une autonomie croissante.\\

\tabletitle{Tableau 2.3}{descriptions des stades}
\begin{center}
\begin{tabularx}{\textwidth}{@{}rX@{}}
\hline
  \textbf{Niveau}
& \textbf{Stade de développement}\\
\hline
  \textbf{Préconventionnel}
& \textbf{1 - punition/récompense}\\
& L'enfant a une obéissance aveugle, il veut éviter la punition jusqu'au point de mentir.\\
& \textbf{2 - moralité instrumentale}\\
& L'enfant cherche à défendre son intéret personnel. Il voit les autres comme des outils.\\
  \textbf{Conventionnel}
& \textbf{3 - moralité des relations interpersonnelles}\\
& La personne cherche l'approbation des autres. Il y a présence d'empathie, car il faut traiter les autres comme pour nous. La bonne action est celle approuvée par le millieu.\\
& \textbf{4 - conscience sociale}\\
& La personne tient compte de l'ensemble de la société. La bonne action respecte l'authorité.\\
  \textbf{Postconventionnel}
& \textbf{5 - contrat social et droits individuels}\\
& La personne est la recherche d'un consensus. La bonne action respecte les droits.\\
& \textbf{6 - conscience sociale}\\
& La personne a une autonomie morale complète. L'action bonne repose sur un système éthique englobant la vie humaine.\\
\hline\\
\end{tabularx}
\end{center}
\end{document}