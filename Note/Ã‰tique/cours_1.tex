\documentclass[11pt]{article}
\usepackage[utf8]{inputenc}
\usepackage[frenchb]{babel}
\usepackage{tabularx}
\usepackage{macro}

\begin{document}
\begin{center}
\begin{tabularx}{\textwidth}{@{}r|X@{}}
  \textbf{Prémisse 1}
& Il existe différentes conceptions du bien et du mal dans différentes cultures.\\
  \textbf{Prémisse 2}
& Il est impossible de savoir avec certitude laquelle de ces conceptions est «la vrai».\\
  \textbf{Conclusion}
& La vérité au sujet du bien et du mal dépend de la culture.\\
\end{tabularx}
\end{center}
La croyance n'est qu'une perception d'une chose, elle n'est pas nécéssairement vrai. Par conséquent, la vérité ne dépend pas de la croyance d'un individue. Les deux prémisses ne peuvent pas être utilisées pour défendre la thèse.
\begin{center}
«être perçu vrai» ou «être vrai» et «être vrai»
\end{center}

\definition{Valeur}{Brièvement, c'est un idéal de vie.}
\definition{Étique}{C'est une étude critique des règles, principes, valeurs et de la morale.}

L'éthique \textbf{normative} est la formulation des règles de conduite pour atteindre la vie bonne. L'éthique \textbf{appliquée} est l'analyse des situations actuelles à la lumière des théories présentes et passées. La \textbf{métaéthique} est la réflexion sur les fondements de l'éthique.

\definition{Morale}{C'est un ensemble des comportements à respecter pour bien agir selon un groupe ou une société.}

\tabletitle{Tableau 1.1}{exemples de groupes où la morale peut différer}
\begin{center}
\begin{tabular}{@{}ll@{}}
\hline
  \textbf{Groupe}
& \textbf{Exemples}\\
\hline
  Culturelle
& occidentale, orientale, etc\\
  Religion
& christianisme, judaisme, etc\\
  Idéologique
& communisme, libertarisme, etc\\
\hline\\
\end{tabular}
\end{center}

\tabletitle{Tableau 1.2}{moralité d'une action}
\begin{center}
\begin{tabular}{@{}ll@{}}
\hline
  \textbf{Type}
& \textbf{Définition}\\
\hline
  Morale
& relève du bien\\
  Immorale
& ne relève pas du bien\\
  Amorale
& ne relève pas du bien ni du mal\\
\hline\\
\end{tabular}
\end{center}

\definition{Philosophie politique}{C'est une étude des questions relatives à la justice dans les sociétés.}
\definition{Question éthique}{C'est une question qui se rapporte toujours à un conflit de normes et/ou de valeurs morales.}

Une question éthique se doit d'être claire par elle-même, mais, à la base, elle doit contenir 4 composantes :
\begin{itemize}
\item[$\bullet$] Elle doit être formulée de manière neutre.
\item[$\bullet$] Elle doit être controvesée.
\item[$\bullet$] Elle doit bien mettre en évidence qu'il y a une controverse relative au bien et au mal.
\item[$\bullet$] Elle doit être binaire.
\end{itemize}
\textbf{Mot clés} : moral, moralement acceptable, éthique, bien, juste, etc

\definition{Norme morale}{C'est une règle qui dicte le comportement à adopter pour agir de façon morale. Une norme morale est négative et impose des limites. \exemple{Il ne faut pas tuer.}}

Un \textbf{principe} est une norme générale et englobante. Se portée est universelle. \exemple{Il faut respecter la propriété privée.} Une \textbf{règle} est une norme plus précise et ciblée.  Sa portée est plus limitée. \exemple{Il faut ramener un livre emprunté.} Par contre, les normes ne sont pas nécéssairement des normes morales. De la même manière, les lois ne sont pas toujours des normes morales valides. \exemple{Il est interdit de jouer au tennis torse nu.}

\definition{Loi}{C'est un règle établie par l'autorité souveraine d'un État et imposée à toute la société.}

La loi n'est pas la Vérité, car les lois dépendent de la location et de l'époque d'une civilisation. Une loi peut être mauvaise ou injuste. De la même manière, un comportement peut être mauvais tout en respectant la loi.

\definition{Valeur}{C'est une croyance durable quant au caractère désirable d'une conduite de vie par rapport à une conduite de vie différente. Donc, une croyance n'est pas indubitable.}

\tabletitle{Tableau 1.3}{exemple de valeurs reliées à des normes}
\begin{center}
\begin{tabular}{@{}rl@{}}
\hline
  \textbf{Valeur}
& \textbf{Norme}\\
\hline
  Honnêteté
& Il ne faut pas mentir.\\
  Santé
& Il ne faut pas fumer.\\
  Vie
& Il ne faut pas tuer.\\
\hline\\
\end{tabular}
\end{center}

\tabletitle{Tableau 1.4}{types de valeurs}
\begin{center}
\begin{tabularx}{\textwidth}{@{}rX@{}}
\hline
  \textbf{Type de valeur}
& \textbf{Définition}\\
\hline
  Vertu
& C'est le trait de caractère d'une personne. Elle définit le type de personne qu'elle voudrait devenir.\\
  Idéal de Vie
& C'est un objectif pour la conduite de la vie d'une personne. Elle définit quelle type de vie la personne voudrait vivre.\\
  Idéal de Société
& C'est un objectif pour la conduite de la société. Elle définit dans quel type de monde une personne voudrait vivre.\\
\hline\\
\end{tabularx}
\end{center}

\tabletitle{Tableau 1.5}{exemple des types de valeurs}
\begin{center}
\begin{tabularx}{\textwidth}{@{}llX@{}}
\hline
  \textbf{Vertus}
& \textbf{Idéaux de Vie}
& \textbf{Idéaux de Société}\\
\hline
  Bienveillance
& Être en forme
& Protection des personnes vulnérables\\
  Générosité
& Dépassement de soi
& Protection de l'envionnement\\
  Honnêteté
& Éducation
& Éducation\\
  Discipline
& Stabilité Financière
& Prosperité\\
  Gentillesse
& Famille
& Égalité\\
\hline
\end{tabularx}
\end{center}
\end{document}