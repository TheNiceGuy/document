\documentclass[11pt]{article}
\usepackage[utf8]{inputenc}
\usepackage[frenchb]{babel}
\usepackage{tabularx}
\usepackage{macro}

\begin{document}
Pour résumer, le devoir est égal à une des trois impératifs catégoriques de Kant. De plus, l'action est meilleur plus elle est fait par devoir. Une action fait pas inclination naturelle (par habitude) est moins bonne qu'une action réfléchit. Il est impossible de faire un mauvais geste avec une bonne intention (suivre le devoir).\\

À l'aide du teste de l'impératif catégorique, il est possible de déterminer si une action est bonne. Si l'action est désinteressé, alors elle est morale. Si l'action est faite avec un intêret, l'action est conforme au devoir.\\

L'éthique de Kant à certains problèmes :
\begin{itemize}
\item[$\bullet$] Elle ne tolère pas les exceptions.
\item[         ] \exemple{Il ne faut pas mentir à un tueur, même si cela pourrait protêger la victime. On ne juge pas l'action en fonction de sa conséquence.}
\item[$\bullet$] Elle ne gère pas les devoirs conflictuels.
\item[         ] \exemple{Il ne faut pas tuer un tueur qui veut nous tuer, même pour de la légitime défense. L'action de tuer est une mauvaise action.}
\item[$\bullet$] Elle est anthropocentrique.
\item[         ] \exemple{Il n'est pas immorale de faire souffir les animaux, car ils ne sont pas digne.}
\item[$\bullet$] Elle a des exigences irréalistes.
\item[         ] \exemple{Il est impossible d'agir de façon totalement désintèressé.}\\
\end{itemize}

L'éthique de Kant à aussi des avantages :
\begin{itemize}
\item[$\bullet$] Elle évite de tenir compte des conséquences.
\item[$\bullet$] Elle accorde une valeur intrinsue à chaque individu.
\end{itemize}
\end{document}