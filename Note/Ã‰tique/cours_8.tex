\documentclass[11pt]{article}
\usepackage[utf8]{inputenc}
\usepackage[frenchb]{babel}
\usepackage{tabularx}
\usepackage{macro}

\begin{document}
Selon l'éthique de Jonas, notre action doit être compatible avec la permanence d'une vie authentiquement humaine sur Terre. De plus, elle définit deux devoirs. 
\begin{itemize}  
\item Il faut connaître les conséquences à long terme de nos actions.
\item Il faut procéder prudemment en envisageant toujours le pire scénario.
\end{itemize}

Cette éthique a plusieurs forces.
\begin{itemize}  
\item Elle n'est pas anthropocentrique.
\item Elle tient compte des évolutions technologiques modernes.
\item Elle a une vision à long terme du monde.
\end{itemize}

Par contre, cette éthique amène certaines problèmes.
\begin{itemize}  
\item C'est une théorie conservatrice. Elle est difficile à mettre en application dans un monde obsédé par la technologie.
\item Elle est imprécise. La vie authentiquement humaine comporte un caractère flou.
\item C'est une théorie paternaliste. Elle risque d'amener à une technocratie, un gouvernement qui est dirigé par les scientifiques.
\end{itemize}
\end{document}