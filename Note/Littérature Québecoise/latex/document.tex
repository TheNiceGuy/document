\documentclass[11pt]{article}
\usepackage[utf8]{inputenc}
\usepackage[T1]{fontenc}
\usepackage{tabularx}
\usepackage{geometry}
\usepackage[neverdecrease]{paralist}
\usepackage{enumitem}
\usepackage{lipsum}
\geometry{
	letterpaper,
	left=10mm,
	right=10mm,
	top=10mm,
	bottom=10mm,
	heightrounded,
}
\setdefaultleftmargin{5mm}{2cm}{}{}{}{}
\setlength{\parindent}{0pt}
\setlist[itemize]{leftmargin=*}
\setlist{nolistsep}
%\tolerance=1
%\emergencystretch=\maxdimen
%\hyphenpenalty=10000
%\hbadness=10000
\pagenumbering{gobble}

\newcommand{\under}[1]{\underline{\smash{#1}}}
\newcommand{\titre}[1]{{\Large\textbf{#1}}}

\begin{document}

\begin{tabularx}{\textwidth}{@{}XX@{}}
  \begin{minipage}[t]{\linewidth\fboxsep\fboxrule}
    \titre{Terroir (1840-1945)}\\
    Cette époque marque le \textbf{fondement de l'identité canadienne-française}. Malgré la défaite des Patriotes et le rapport Durham, les Canadiens français restent enracinés à leur culture distincte (religion, valeurs, langue, contes, etc). L'idéologie de la conservation et du repli se développe et elle est entretenu par l'Église.
    \begin{itemize}
      \item\textbf{Thèmes:} identité du peuple, la religion, la famille, le travail de la terre
      \item\textbf{Personnages:} le paysan, le coureur des bois, le curé, la femme, la religieuse, l'anglophone, les amérindiens
    \end{itemize}
    \textbf{Le roman idéologique:} Il est marqué par une idéalisation de la terre et par la conservation des valeurs (ex: \textit{Maria Chapedelaine} de Louis Hémon). \textbf{Le roman naturaliste:} Il est a l'opposé du roman idéologique. Il montre les aspects négatifs du terroir (ex: \textit{La Scouine} de Albert Laberge, \textit{Un homme et son péché} de Claude-Henri Grignon). \textbf{Le roman réaliste:} Il annonce un changement, un détachement, une rupture du terroir (ex: \textit{Le Survenant} de Germaine Guèvremont).\\
  \end{minipage}&
  
  \begin{minipage}[t]{\linewidth\fboxsep\fboxrule}
    \titre{Littérature Urbaine (1945-1960)}\\
    Cette époque \textbf{menace l'identité canadienne-française} développée durant l'époque du terroir. La crise économique de 1929 et la seconde guerre mondiale influencent les auteurs de cet époque. Le phénomène de l'urbanisation et de l'industralisation entraîne un combat entre la conservation des valeurs versus la modernité. L'église cherche à maintenir son pouvoir dans cette société qui appelle le changement.
    \begin{itemize}
      \item\textbf{Thèmes:} la conservation, le déracinement et la dépossession, le travail, l'aliénation, la solitude, l'intimité
      \item\textbf{Personnages:} l'ouvrier, la femme indépendante, le curé, l'intellectuel marginalisé, l'étranger, les soldats, le père 
    \end{itemize}
    Il y a 3 tendances majeures dans cette littérature: \textbf{le réalisme} (ex: \textit{Un simple soldat} de Roger Lemelin, \textit{Bonheur d'occasion} de Gabrielle Roy), \textbf{le surréalisme} et \textbf{l'existentialisme} (ex: \textit{L'étranger} de Albert Camus).\\
  \end{minipage}\\
  
  \begin{minipage}[t]{\linewidth\fboxsep\fboxrule}
    \titre{Âge de la parole (1960-1980)}\\
    Cette époque montre l'\textbf{affirmation de l'identité québecoise}. C'est la fin de la Grande Noirceur, le règne de Duplessis. Après des années d'aliénation par les anglais, le terme « Québecois » et l'idée d'un projet collectif d'indépendance apparaissent. L'église perd son pouvoir religueux au gouvernement qui prend le rôle de l'État providence. Le Québec s'ouvre au monde avec l'Expo '67. La mondialisation remet en question des valeurs capitalistes, la censure, les minorités (féminisme, homosexuel), décolonisation, etc. 
    \begin{itemize}
      \item\textbf{Thèmes:} la famille, la sexualité (le corps), la religion, l'aliénation, l'identité, le pays, la langue, la parôle (affirmation identitaire), le travail (dénonciation de l'exploitation)
      \item\textbf{Personnages:} le peuple, l'intellectuel porte-parole, les marginaux (travestis, homoxesuels, etc), la famille éclatée, la femme (féminisme)
    \end{itemize}
    \textbf{Le misérabilisme:} Il est tourné vers le passé et il dénonce la misère, la pauvreté et l'aliénation vécu par les québecois. Le joual est souvent employé comme language (ex: \textit{Les Belles-soeurs} de Michel Tremblay). \textbf{La littérature engagée:} Elle est tournée vers l'avenir et elle appel la liberté et met en contexte l'espoir d'un pays (ex: \textit{Gens du pays} de Gilles Vigneault). \textbf{Le féminisme:} Il dénonce l'inégalité entre les femmes et les hommes. Les femmes deviennent indépendantes et elles s'attaquent à la langue, aux préjugés, au corps et à la sexualité. \textbf{La contre-culture:} Elle est une expérimentation. Elle montre l'éclatement des valeurs, la marginalité et elle critique le capitalisme.\\
  \end{minipage}&
  
  \begin{minipage}[t]{\linewidth\fboxsep\fboxrule}
    \titre{Postmodernité (1980 à aujourd'huie)}\\
    Cette époque marque le commencement d'\textbf{une mutation à l'identité québecoise}. Les grandes idéologies comme le nationalisme, le féminisme et le socialisme sont remplacées par l'individualisme, le néo-libéralisme et l'altermondialisme. L'immigration, la mondialisation et l'ère des communications changent notre rapport au monde.
    \begin{itemize}
      \item\textbf{Thèmes:} l'éclatement des valeurs (éclatement de la famille et du couple, dissolution, reconstitution, dédoublement, dérive, perte des repères, absence de modèle, etc), l'individualisme au néo-libéralisme (solitude, illusion d'appartenance, désenchantement, dérive identitaire, hédonisme, consommation, matérialisme, etc), la présence de l'immigrant (hybridation, choc des cultures, accomodements dé/raisonnables, etc) et le développement des technologies (le virtuel, l'imaginaire, l'autofiction, etc). 
      \item\textbf{Personnages:} l'étranger, l'immigrant, l'écrivain, l'artiste, l'enfant, l'adolescent
    \end{itemize}
    La littérature est beaucoup marquée par les ruptures continuelles du temps (ex: \textit{Le chemin des passes dangereuses} de Michel-Marc Bouchard) et de la narration afin de perdre le lecteur (ex: \textit{Il pleuvait des oiseaux} de Jocelyne Saucier). Elle est aussi formée par mélange de genres, de tonalitées et de style (hybridation). Finalement, l'intertextualité est souvent employé.\\
  \end{minipage}\\
\end{tabularx}
\pagebreak

\begin{tabularx}{\textwidth}{@{}XX@{}}
  \begin{minipage}[t]{\linewidth\fboxsep\fboxrule}
    \titre{Structure de la dissertation}\\
    \textbf{Introduction}
    \begin{enumerate}
      \item\textit{Sujet amené}: contexte historique (époque, années, évènements, etc).
      \item\textit{Sujet posé}: le nom de l'auteur, le nom de l'oeuvre, le genre littéraire (poésie, roman, pièce de théâtre, etc), le courant littéraire et une courte description de l'extrait.
      \item\textit{Sujet divisé}: prise de position et les trois aspects abordés dans la dissertation.
    \end{enumerate}

    \textbf{Structure d'un paragraphe}
    \begin{enumerate}
      \item Le paragraphe commence normalement par un marqueur de relation, sauf si c'est le premier.
      \item On annonce l'idée principale du paragraphe.
      \item On aborde les deux idées secondaires avec un illustration (passage de l'extrait étudié) et une explication.
      \item On finit le paragraphe avec une conclusion partielle (un petit résumé, sans se répeter).
    \end{enumerate}
    
    \textbf{L'explication}
    \begin{enumerate}
      \item On identifie le procédé stylistique.
      \item On établie le lien avec l'idée secondaire.
      \item On ramène cette idée à l'idée principale.
    \end{enumerate}
    Il est idéale de faire le plus de liens possibles avec le contexte social et littéraire.

    \textbf{Conclusion}
    \begin{enumerate}
      \item\textit{Rappel de la thèse}
      \item\textit{Synthèse}: ce que l'on retient de l'analyse (ne pas être répétitif)
      \item\textit{Ouverture}: lien justifié avec une autre oeuvre, exploiter le contexte historique, ouvrir sur l'oeuvre complète ou réflexion personnelle.\\
    \end{enumerate}
  \end{minipage}
  
  \begin{minipage}[t]{\linewidth\fboxsep\fboxrule}
    \titre{Marqueur de relation}
    \begin{itemize}
      \item\textbf{Cause:} \textit{car}, \textit{en effet}, \textit{en raison de}
      \item\textbf{Conséquence:} \textit{ainsi}, \textit{c'est pourquoi}, \textit{donc}, \textit{en conséquence}, \textit{de là}, \textit{d'où}
      \item\textbf{Addition:} \textit{ainsi que}, \textit{bien plus}, \textit{de plus}, \textit{et}, \textit{puis}
      \item\textbf{Opposition:} \textit{au contraire}, \textit{cependant}, \textit{en revanche}, \textit{mais}, \textit{par contre}, \textit{pourtant}, \textit{toutefois}, \textit{plutôt}, \textit{même si}, \textit{malgré}
      \item\textbf{Similiture:} \textit{pareillement}, \textit{de manière semblable}, \textit{dans le même ordre d'idées}, \textit{aussi}
      \item\textbf{But:} \textit{afin de}, \textit{à seule fin de}, \textit{assez pour}, \textit{en vue de}\\
    \end{itemize}
  \end{minipage}
  
  \begin{minipage}[t]{\linewidth\fboxsep\fboxrule}
    \titre{Verbes introducteurs de citations}\\
    \textit{souligner}, \textit{illustrer}, \textit{représenter}, \textit{renvoyer à}, \textit{faire référence}, \textit{mettre en lumière}, \textit{rappeller le thème}, \textit{évoquer}, \textit{mettre de l'avant}, \textit{mettre au jour}, \textit{mettre en évidence}, \textit{montre}, \textit{traduire l'idée que}
  \end{minipage}&
  
  \begin{minipage}[t]{\linewidth\fboxsep\fboxrule}
    \titre{Figures de style}
    \begin{itemize}
      \item\textbf{Sobriété du style:} phrases courtes, style neutre, abscence de figure de style, description objective, passage dénué d'émotions.
      \item\textbf{Narration à la première personne:} accès aux sentiments et aux sensations du personnage (ou l'absence).
      \item\textbf{Niveau de langue:} standart, familier, le personnage semble ordinaire (ou non).
      \item\textbf{Hyperbole:} exagération, produit une forte impression.
      \item\textbf{Accumulation:} énumération de mots, de phrases ou de prépositions. Elle crée un effet de profusion.
      \item\textbf{Gradation:} progression qui crée un effet de dramatisation.
      \item\textbf{Euphémisme:} atténue le sens d'une réalité ou d'un mot trop cru en le remplaçant par une formulation moins brutale.
      \item\textbf{Litote:} consiste à en dire moins pour en faire entendre d'avantage. Elle donne plus de force à une assertion en paraissant l'atténuer.
      \item\textbf{Personnification:} attribuer un caractère humain à un objet ou un animal. Elle rend concrète une notion abstraite.
      \item\textbf{Depersonnification:} attribuer un caractère animal à un être humain.
      \item\textbf{Répétition}, \textbf{Pléonasme}
      \item\textbf{Antithèse}, \textbf{Oxymore}
      \item\textbf{Comparaison}, \textbf{Métaphore}
      \item\textbf{Autres:} tonalité philosophique, temps de verbes, ponctuations, l'intertextualité\\
    \end{itemize}
  \end{minipage}  
  
  \begin{minipage}[t]{\linewidth\fboxsep\fboxrule}
    \titre{Problèmes fréquents d'une dissertation}
    \begin{itemize} 
      \item Il faut s'assurer de bienc comprendre l'énoncé et de rechercher les mots clés dans le dictionnaire afin d'obtenir plus d'idées pour l'analyse.
      \item Il est préférable de ne pas prendre le mot important de l'énoncé dans l'argumentation afin de ne pas être répétitif et de développer plus l'analyse.
      \item Il faut faire plus possible de liens avec le contexte ou le courant littéraire.
      \item Il ne faut pas poser une question rhétorique dans l'ouverture de la conclusion sans y répondre avec une petite réflexion.
    \end{itemize}
  \end{minipage}\\
  
\end{tabularx}
\end{document}
