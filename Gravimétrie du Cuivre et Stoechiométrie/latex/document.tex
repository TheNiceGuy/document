\documentclass[11pt]{article}
\usepackage[T1]{fontenc}
\usepackage[utf8]{inputenc}
\usepackage{gensymb}
\usepackage{mathtools}
\usepackage[version=4]{mhchem}
\usepackage{tabularx}
\usepackage{geometry}
\usepackage[table]{xcolor}
\usepackage{multirow}
\usepackage{siunitx}
\usepackage{textcomp}
\usepackage{ragged2e}
\usepackage{lettrine}
\usepackage{lmodern}
\usepackage{fp}
%\usepackage{mathptmx}
\tolerance=1
\emergencystretch=\maxdimen
\hyphenpenalty=10000
\mathtoolsset{showonlyrefs}
\sisetup{output-decimal-marker={,}}
\newcolumntype{C}{>{\centering\arraybackslash}X}
\newcolumntype{A}{X}
\newcolumntype{a}{@{}l@{}}
\geometry{
	letterpaper,
	left=42mm,
	right=42mm,
	top=42mm,
	bottom=42mm,
	heightrounded,
}

\newcommand{\g}{\ensuremath{\,\textrm{g}}}
\newcommand{\gmol}{\ensuremath{\,\textrm{g/mol}}}
\newcommand{\ml}{\ensuremath{\,\textrm{ml}}}
\newcommand{\mol}{\ensuremath{\,\textrm{mol}}}
\newcommand{\mmol}{\ensuremath{\,\textrm{mmol}}}
\newcommand{\M}{\ensuremath{\,\textrm{mol/L}}}
\newcommand{\percent}{\ensuremath{\,\textrm{\%}}}
\newcommand{\ps}{\vspace{.2\baselineskip}}

\newenvironment{nscenter}
	{\parskip=0pt\par\nopagebreak\centering}
	{\par\noindent\ignorespacesafterend}

%%%%%%%%%%%%%%%%%%%%%%%%%
% CALCULS MATHÉMATIQUES %
%%%%%%%%%%%%%%%%%%%%%%%%%
\def\MasseMolaireZn{65.38}%g/mol
\def\MasseMolaireCu{63.546}%g/mol
\def\LimitantMasseZn{0.124}%g
\def\LimitantMasseMontage{90.109}%g
\def\LimitantMasseMontageCu{90.212}%g
\def\ExcesMasseZn{0.528}%g
\def\ExcesMasseMontage{150.639}%g
\def\ExcesMasseMontageCu{150.950}%g FIXME
\def\SolutionVolume{50.00}%ml
\def\SolutionConcentration{0.150}%mol/L

\FPeval{\QuantiteSolution}{round(\SolutionVolume*\SolutionConcentration,2)}%mmol
\FPeval{\LimitantMasseCu}{round(\LimitantMasseMontageCu-\LimitantMasseMontage,3)}%g
\FPeval{\LimitantQuantiteZn}{round(\LimitantMasseZn/\MasseMolaireZn*1000,2)}%mmol
\FPeval{\LimitantQuantiteCuTheorique}{round(\LimitantQuantiteZn,2)}%mmol
\FPeval{\LimitantQuantiteCuExperimental}{round(\LimitantMasseCu/\MasseMolaireCu*1000,2)}%mmol
\FPeval{\LimitantRendement}{round(\LimitantQuantiteCuExperimental/\LimitantQuantiteCuTheorique*100,1)}
\FPeval{\ExcesMasseCu}{round(\ExcesMasseMontageCu-\ExcesMasseMontage,3)}%g
\FPeval{\ExcesQuantiteZn}{round(\ExcesMasseZn/\MasseMolaireZn*1000,2)}%mmol
\FPeval{\ExcesQuantiteCuTheorique}{round(\QuantiteSolution,2)}%mmol
\FPeval{\ExcesQuantiteCuExperimental}{round(\ExcesMasseCu/\MasseMolaireCu*1000,2)}%mmol
\FPeval{\ExcesRendement}{round(\ExcesQuantiteCuExperimental/\ExcesQuantiteCuTheorique*100,1)}

\begin{document}
\begin{titlepage}
\begin{center}
	{\large Rapport de Laboratoire}\\
	{\huge Champ $\vec{E}$ et Équipotentielles}
	\\[38mm]
	Olivier Besner\\
	Rosalie Lapointe\\
	Mohamed Amine\\
	Gabriel-Andrew Pollo-Guilbert
	\\[38mm]
	Électricité et Magnétisme\\
	Physique 203-NYB-05
	\\[38mm]
	Travail présenté à\\
	David Lambert
	\\[38mm]
	Sciences de la Nature\\
	Sciences Informatiques et Mathématiques\\
	14 Juillet 2015\\
\end{center}
\end{titlepage}

%%%%%%%%%%%%%%%%
% INTRODUCTION %
%%%%%%%%%%%%%%%%
La\\


\section*{Mesures et résultats}
%%%%%%%%%%%%%%%%%%%%%%%
% TABLEAU DES MESURES %
%%%%%%%%%%%%%%%%%%%%%%%
\vspace{-3mm}\noindent\setlength\tabcolsep{0pt}
\begin{center}
\begin{tabularx}{{\textwidth}}{lX}
	  \textbf{Tableau des mesures : }
	& masses de \ce{Zn} et du montage incluant le \ce{Cu}\\
\end{tabularx}
\begin{tabularx}{{\textwidth}}{@{}|XCX|XS[table-format=1.2]X|XS[table-format=1.2]X|@{}}
\hline
	&&& \multicolumn{3}{c|}{Filtration par gravité}
	&   \multicolumn{3}{c|}{Filtration sous vide}\\
	&&& \multicolumn{3}{c|}{\textbf{Zinc limitant}}
	&   \multicolumn{3}{c|}{\textbf{Zinc en excès}}\\
\hline
	    \multicolumn{3}{|c|}{\parbox[c]{5.5cm}{\ps\centering \ce{Zn}\\(g)\ps}}
	&&  \LimitantMasseZn
	&&& \ExcesMasseZn&\\
\hline
	    \multicolumn{3}{|c|}{\parbox[c]{5.5cm}{\ps\centering Montage\\(g)\ps}}
	&&  \LimitantMasseMontage
	&&& \ExcesMasseMontage&\\
\hline
	    \multicolumn{3}{|c|}{\parbox[c]{5.5cm}{\ps\centering Montage+\ce{Cu}\\(g)\ps}}
	&&  \LimitantMasseMontageCu
	&&& \ExcesMasseMontageCu&\\
\hline
\end{tabularx}
%\begin{tabularx}{{\textwidth}}{@{}|XCX|XS[table-format=3.3]X|@{}}
%\hline
%	&&& \multicolumn{3}{ c|}{Masse}\\
%	    \multicolumn{3}{|c|}{Matériel}
%	&   \multicolumn{3}{ c|}{(g)}\\
%	&&& \multicolumn{3}{ c|}{$\pm0,001$}\\
%\hline
%       \multicolumn{6}{|c|}{Filtration par gravité}\\
%       \multicolumn{6}{|c|}{\textbf{Zinc limitant}}\\
%\hline
%	   \multicolumn{3}{|c|}{\ce{Zn}}
%	&& \LimitantMasseZn&\\
%\hline
%	   \multicolumn{3}{|c|}{Papier filtre+Verre de montre}
%	&& \LimitantMasseMontage&\\
%\hline
%	   \multicolumn{3}{|c|}{Papier filtre+Verre de montre+\ce{Cu}}
%	&& \LimitantMasseMontageCu&\\
%\hline
%       \multicolumn{6}{|c|}{Filtration sous vide}\\
%       \multicolumn{6}{|c|}{\textbf{Zinc en excès}}\\
%\hline
%	   \multicolumn{3}{|c|}{\ce{Zn}}
%	&& \ExcesMasseZn&\\
%\hline
%	   \multicolumn{3}{|c|}{Papier filtre+Büchner+Capsule}
%	&& \ExcesMasseMontage&\\
%\hline
%	   \multicolumn{3}{|c|}{Papier filtre+Büchner+Capsule+\ce{Cu}}
%	&& \ExcesMasseMontageCu&\\
%\hline
%\end{tabularx}
\justifying\textbf{Volume de la solution de \ce{CuSO4} :} $(\num\SolutionVolume\pm0,05)\ml$\\
\justifying\textbf{Concentration de la solution de \ce{CuSO4} :} $(\num\SolutionConcentration\pm0,001)\M$
\end{center}

%%%%%%%%%%%%%%%%%%%%%%%%%
% TABLEAU DES RÉSULTATS %
%%%%%%%%%%%%%%%%%%%%%%%%%
\noindent\setlength\tabcolsep{0pt}
\begin{center}
\begin{tabularx}{{\textwidth}}{lX}
	  \textbf{Tableau des résultats : }
	& quantités de \ce{Zn}, \ce{CuSO4} et \ce{Cu}, et \% de rendement\\
\end{tabularx}
\begin{tabularx}{{\textwidth}}{@{}|XCX|XS[table-format=1.2]X|XS[table-format=1.2]X|@{}}
\hline
	&&& \multicolumn{3}{c|}{Filtration par gravité}
	&   \multicolumn{3}{c|}{Filtration sous vide}\\
	&&& \multicolumn{3}{c|}{\textbf{Zinc limitant}}
	&   \multicolumn{3}{c|}{\textbf{Zinc en excès}}\\
\hline
	    \multicolumn{3}{|c|}{\parbox[c]{5.5cm}{\ps\centering Quantité de \ce{Zn}\\(mmol)\ps}}
	&&  \LimitantQuantiteZn
	&&& \ExcesQuantiteZn&\\
\hline
	    \multicolumn{3}{|c|}{\parbox[c]{5.5cm}{\ps\centering Quantité de \ce{CuSO4}\\(mmol)\ps}}
	&   \multicolumn{6}{ c|}{\QuantiteSolution}\\
\hline
\hline
	    \multicolumn{3}{|c|}{\parbox[c]{5.5cm}{\ps\centering Quantité de \ce{Cu} théorique\\(mmol)\ps}}
	&&  \LimitantQuantiteCuTheorique
	&&& \ExcesQuantiteCuTheorique&\\
\hline
	    \multicolumn{3}{|c|}{\parbox[c]{5.5cm}{\ps\centering Quantité de \ce{Cu} expérimentale\\(mmol)\ps}}
	&&  \LimitantQuantiteCuExperimental
	&&& \ExcesQuantiteCuExperimental&\\
\hline
\hline
	    \multicolumn{3}{|c|}{\% de rendement}
	&&  \parbox[c]{3cm}{\centering \num\LimitantRendement}
	&&& \parbox[c]{3cm}{\centering \num\ExcesRendement}&\\
\hline
\end{tabularx}
\end{center}

\section*{Calculs}
%%%%%%%%%%%%%%%%%%%%%%%
% CALCULS - PARTIE SV %
%%%%%%%%%%%%%%%%%%%%%%%
\subsection*{Filtration sous vide}
\begin{enumerate}
{\bfseries\item Nombre de mol de \ce{Zn}}
\begin{equation*}
\begin{split}
\mathrm{n_{\ce{Zn}}}&=\frac{\mathrm{m_{\ce{Zn}}}}{\mathrm{MM_{\ce{Zn}}}}\\
	                &=\frac{\num\ExcesMasseZn\g}{\num\MasseMolaireZn\gmol}\\
	                &=\num\ExcesQuantiteZn\mmol\\
\end{split}
\end{equation*}

{\bfseries\item Nombre de mol de \ce{CuSO4}}
\begin{equation*}
\begin{split}
\mathrm{n_{\ce{CuSO4}}}&=\mathrm{V_{\ce{CuSO4}}\cdot C_{\ce{CuSO4}}}\\
	                   &=\num\SolutionVolume\ml\cdot \num\SolutionConcentration\M\\
	                   &=\num\QuantiteSolution\mmol\\
\end{split}
\end{equation*}

{\bfseries\item Nombre de mol de \ce{Cu} théorique produite par le \ce{Zn}}
\begin{equation*}
\begin{split}
\mathrm{n_{\ce{Cu}}}&=\mathrm{n_{\ce{Zn}}}\cdot\frac{\mathrm{coef_{\ce{Cu}}}}{\mathrm{coef_{\ce{Zn}}}}\\
	                &=\num\ExcesQuantiteZn\mmol\cdot\frac{1\mol}{1\mol}\\
	                &=\num\ExcesQuantiteZn\mmol\\
\end{split}
\end{equation*}

{\bfseries\item Nombre de mol de \ce{Cu} théorique produite par le \ce{CuSO4}}
\begin{equation*}
\begin{split}
\mathrm{n_{\ce{Cu}}}&=\mathrm{n_{\ce{CuSO4}}}\cdot\frac{\mathrm{coef_{\ce{Cu}}}}{\mathrm{coef_{\ce{CuSO4}}}}\\
	                &=\num\QuantiteSolution\mmol\cdot\frac{1\mol}{1\mol}\\
	                &=\num\QuantiteSolution\mmol\\
\end{split}
\end{equation*}

{\bfseries\item Masse de \ce{Cu} expérimentale}
\begin{equation*}
\begin{split}
\mathrm{m_{\ce{Cu}}}&=\mathrm{m_{\text{papier filtre+büchner+capsule+\ce{Cu}}}
                     -\mathrm{m_{\text{papier filtre+büchner+capsule}}}}\\
	                &=\num\ExcesMasseMontageCu\g-\num\ExcesMasseMontage\g\\
	                &=\num\ExcesMasseCu\g\\
\end{split}
\end{equation*}

{\bfseries\item Nombre de mol de \ce{Cu} expérimental}
\begin{equation*}
\begin{split}
\mathrm{n_{\ce{Cu}}}&=\frac{\mathrm{m_{\ce{Cu}}}}{\mathrm{MM_{\ce{Cu}}}}\\
	                &=\frac{\num\ExcesMasseCu\g}{\num\MasseMolaireCu\gmol}\\
	                &=\num\ExcesQuantiteCuExperimental\mmol\\
\end{split}
\end{equation*}

{\bfseries\item Rendement de la réaction}
\begin{equation*}
\begin{split}
\textrm{\% de rendement}&=\frac{\mathrm{n_{\text{\ce{Cu} expérimental}}}}
                               {\mathrm{n_{\text{\ce{Cu} théorique}}}}
                               \times 100\\
                        &=\frac{\num\ExcesQuantiteCuExperimental\mmol}{\num\ExcesQuantiteCuTheorique\mmol}
                               \times 100\\
                        &=\num\ExcesRendement\percent
\end{split}
\end{equation*}

\end{enumerate}

%%%%%%%%%%%%%%
% DISCUSSION %
%%%%%%%%%%%%%%
\section*{Discussion}


%%%%%%%%%%%%%%
% CONCLUSION %
%%%%%%%%%%%%%%
\section*{Conclusion}
\end{document}
