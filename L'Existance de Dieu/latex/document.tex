\documentclass{article}
\usepackage[T1]{fontenc}
\usepackage[utf8]{inputenc}
\usepackage[finnish]{babel}
\usepackage{marginnote}
\usepackage{setspace}
\usepackage{tikz}
\usepackage{geometry}
\geometry{
	total={210mm,297mm},
	left=1.70in,
	right=1.70in,
	top=1.70in,
	bottom=1.70in,
}

\newcommand{\note}[1]{{
	\marginpar{
		\raggedright
		\footnotesize
		\hfill
		\setstretch{1.025}%
		#1
}}}

\usepackage{amsmath}
\begin{document}
\pagenumbering{gobble}
\noindent Gabriel-Andrew Pollo Guilbert \hfill 5 Mai 2015
\begin{center}
	\huge{L'Existence de Dieu}
\end{center}
Avec l'augmentation de l'immigration au Québec, certaines valeurs sont remises en question. Entre autres, l'enjeu du port du voile en public rammène à la question de la laïcité dans notre province. En effet, ces différentes cultures demandent une certaine liberté dans la pratique de leur religion. D'un point de*\note{50 mots}vue objectif, on peut se demander où devrait être faite la ligne entre la religion et la société, mais il est très difficile de répondre à cette question, car il existe plusieurs croyances avec des requêtes qui diffèrent. Comment peut-on savoir laquelle à raison? Elles croient tout en un dieu*\note{100 mots}différent. Cela emmène une question: est-ce que Dieu existe vraiment, car il est vu différemment d'une religion à une autre? Durant la Renaissance, un mouvement philosophique réapparaît: le scepticisme. Cette doctrine peut se résumer au fait que rien ne peut être prouvé avec certitude. Les sceptiques remettent tout en question,*\note{150 mots}dont l'existence de Dieu. Pendant une expérience de pensée ayant pour but de combattre ce mouvement, René Descartes, un philosophe français de cette époque, tente de prouver que Dieu existe bel et bien. Quelques siècles plus tard, un médecin nommé Freud s'intéresse aux patients hystériques. À cette époque, il était*\note{200 mots}très difficile de guérir les personnes souffrantes de maladies mentales. Le médecin établit alors une théorie révolutionnante sur la conscience humaine à laquelle Dieu n'existerait pas nécessairement.\\

Pour commencer, Descartes trouve trois évidences de l'existence de Dieu. Parmi l'une d'elles, il utilise la définition de Dieu afin de prouver son existence.*\note{250 mots}Dieu peut être défini comme une entité parfaite et infinie. Le philosophe, étant imparfait et limité, dit que l'idée de la perfection ne peut pas venir de lui, car il n'est pas parfait lui-même. Il va jusqu'à dire «Il y a une seule idée en moi que je ne suis*\note{300 mots}pas assez parfait pour avoir produite: Dieu.» Si elle ne provient pas de lui, alors d'où provient-elle? Il conclut qu'elle provient de Dieu lui même. Seulement lui, en perfection, peut créer l'idée de la perfection. Il doit donc exister pour forger une telle pensée. Une autre preuve de l'auteur est*\note{350 mots}similaire à la dernière. Si l'idée de Dieu est la perfection et il n'existe pas, il n'est alors pas parfait. La perfection implique qu'il existe et aussi qu'il est éternel. S'il avait un début ou une fin, on peut supposer qu'il n'aurait pas toujours existé, ce qui ferait de lui*\note{400 mots}un être imparfait. Bref, Dieu existe, selon Descartes, et il serait parfait, illimité et éternel.\\

Ensuite, Freud émet l'idée que Dieu n'existe pas dans la théorie sur la psyché humaine. Dans son hypothèse, l'esprit humain peut être séparé en trois sections: le Ça, le Moi et le Surmoi. La première est*\note{450 mots}la partie inconsciente du cerveau humain. Celle-ci ne se contrôle pas et ne tient qu'à réaliser des désirs enfouis dans l'âme. Ensuite, le Moi peut être vue comme la partie plus conscience et analytique de la raison. Il lie l'inconscience au monde extérieur, alors il permet de contrôler les pulsions*\note{500 mots}de l'organisme. Finalement, le Surmoi est comme le juge de l'esprit. Il est perfectionniste et amène le principe de la morale dans le cerveau. Donc, il émet le sentiment de la culpabilité chez l'homme. La question de l'existence de Dieu règne nécessairement dans la dernière partie: le Surmoi. Comme mentionné*\note{550 mots}plus haut, Dieu est un être infini et parfait. Mais selon Freud, ces concepts ne résident que dans le Surmoi, donc ils ne sont que des projections idéales que le Moi tente d'atteindre. Illustrons le Surmoi avec un passage de la bible: «Tu aimeras ton prochain comme toi-même.» (Marc 12:31)*\note{600 mots}Cela implique qu'il ne faudrait pas faire de mal à une personne, mais Freud dirait que cette pensée provient du Surmoi. Idéalement, on ne voudrait pas faire de mal à une personne parce que l'on ne voudrait pas avoir du mal en retour. Le Surmoi émet une contrainte morale à*\note{650 mots}la psyché humaine. Pour résumer, il n'y a aucune trace de Dieu chez Freud, car il n'est qu'une image idéale du Surmoi.\\

Pour continuer, on peut voir certaines ressemblances et différences chez les deux auteurs. Bien sûr, l'existence de Dieu diffère de Descartes à Freud. Par contre, les deux ont recourent*\note{700 mots}à la conscience humaine pour prouver leur thèse. En effet, Descartes emprunte ce chemin lorsqu'il tente de déterminer la cause de l'idée de la perfection. Il cherche jusqu'à n'en conclure que cette idée provient de Dieu lui-même. Il est intéressant d'analyser Freud en même temps, car il saurait répondre à*\note{750 mots}Descartes. Il trouve la provenance de l'idée que Descartes cherchait dans le Surmoi. Cette partie de la conscience humaine laisse une image de la perfection qui ne viendrait pas de Dieu, mais d'une identification au monde extérieur où s'est développée l'âme. Cela peut aussi expliquer pourquoi la perfection d'un concept*\note{800 mots}distinct peut différer d'une personne à une autre. Effectivement, la vision d'un monde parfait d'un américain peut être différente de celle d'un arabe. Bref, Dieu n'a pas la même place chez les deux auteurs, mais ils ont tout de même des arguments en utilisant le même objet: la conscience.\\

À mon*\note{850 mots}avis, Descartes à raison en disant que Dieu existe, car il nous aurait donné les concepts de l'infini et de la perfection. Certains disent que l'on peut découvrir une notion en utilisant le contraire d'une définition d'un premier concept. Si l'on s'imagine un homme ayant vécu à une même température*\note{900 mots}chaleureuse toute sa vie, est-ce qu'il comprendrait le concept du froid? Je suis porté à dire non, car il faudrait qu'il ressente cette fraicheur ou que quelqu'un tente de lui expliquer cette notion. Cet exemple est similaire à celui de l'infini ou de la perfection. Comment pouvons-nous comprendre ces concepts*\note{950 mots}lorsque nous avons vécu toute notre vie dans le fini et l'imparfait? Ils doivent provenir d'ailleurs, de Dieu lui-même, car il détient ces concepts en lui.\\

En conclusion, l'existence de Dieu est une question intéressante à regarder d'un point de vue philosophique. De grands philosophes ont réfléchis à cette question. Descartes*\note{1000 mots}pense qu'il existe, car l'idée de la perfection ne peut pas provenir de lui-même. Par contre, Freud est porté à croire qu'il n'existe pas, car la perfection n'est qu'une projection du Surmoi dans la conscience. Le cerveau humain est une machine extrêmement complexe. En revanche, les avancées technologiques dans le*\note{1050 mots}domaine de la neurologie nous permettent de mieux comprendre le fonctionnement de l'âme, une intangibilité maintenant à notre portée. Il serait intrigant d'introduire la science moderne dans les hypothèses de Descartes et Freud*\note{1085 mots}.
\end{document}